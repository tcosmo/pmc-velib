\documentclass[a4paper, 11pt]{article}
 
\usepackage{ucs}
\usepackage[utf8x]{inputenc}
\usepackage{graphicx}
\usepackage[frenchb]{babel}
 
\begin{document}
 
\title{Dossier de TIPE}
\author{Camille Chanial, Tristan Du Castel, \text{Tristan Stérin}}
\date{05/06/2014} 
 
\maketitle

\abstract{caca} 
 
\tableofcontents
 

\newpage

\section{Introduction}
\section{La justification mathématiques}
\section{L'algorithme de rétropropagation}

On a donc vu que les réseaux de neurones présentent un modèle pertinent dans le cadre de la régression non linéaire.\\
Il faut désormais mettre en place l'algorithme qui va permettre de trouver, d'apprendre, la combinaison de poids optimales afin de réaliser l'approximation désirée.\\
On se donne une base d'apprentissage :

$$\matcal{B} = \{(p_1,q_1), \dots, (p_n, q_n)\} \quad \text{ avec : } \quad (p_i,q_i) \in \matcal{A}^a \times \matcal{B}^b$$

Où $\matcal{A}$ et $\matcal{B}$ sont les espaces de départ et d'arrivée et $a$ et $b$ leur dimension respective.\\
Si on note $f:\matcal{A}\to \matcal{B}$ la fonction du réseau, notre but est d'avoir en un nombre fini d'itérations :

$$ \forall i, \quad f(p_i) = q_i $$





\section{L'application pratique : aspects techniques et résultats}

 
\end{document}